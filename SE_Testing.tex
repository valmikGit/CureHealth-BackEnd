\documentclass[12pt]{article}
\usepackage[a4paper,margin=1in]{geometry}
\usepackage{titlesec}
\usepackage{hyperref}
\usepackage{enumitem}
\usepackage{graphicx}

\title{Detailed Report on API Testing Code}
\author{}
\date{}

\begin{document}

\maketitle

\section*{1. Framework and Tools Used}

\subsection*{APITestCase}
\begin{itemize}
    \item A subclass of \texttt{unittest.TestCase}, provided by Django REST Framework (DRF) to test API endpoints.
    \item Sets up a test client (\texttt{APIClient}) to simulate HTTP requests and responses.
    \item Ensures each test runs in isolation with an independent database.
\end{itemize}

\subsection*{APIClient}
\begin{itemize}
    \item A client that sends requests to API endpoints and retrieves responses.
    \item Supports HTTP methods like GET, POST, PUT, PATCH, and DELETE.
\end{itemize}

\subsection*{Database Setup}
\begin{itemize}
    \item The database is reset after each test case execution.
    \item Sample data is created in the \texttt{setUp()} method for testing purposes.
\end{itemize}

\section*{2. Explanation of Test Cases}

\subsection*{a. \texttt{PatientsAPITestCase}}
This test case validates the functionality of patient-related endpoints.

\textbf{Setup:}
\begin{itemize}
    \item A \texttt{Patient} object is created with attributes such as \texttt{email}, \texttt{username}, \texttt{blood group}, and \texttt{disease}.
    \item The \texttt{set\_password()} method ensures the password is hashed before saving.
\end{itemize}

\textbf{Tests:}
\begin{itemize}
    \item \texttt{test\_get\_all\_patients}: Simulates a GET request to retrieve all patients and validates the response.
    \item \texttt{test\_get\_patient\_by\_id}: Fetches a specific patient by their ID and checks attributes.
    \item \texttt{test\_get\_patient\_not\_found}: Simulates a GET request with a non-existent ID and expects an error message.
    \item \texttt{test\_get\_patient\_wrong\_type}: Tests querying a \texttt{Doctor} object instead of \texttt{Patient}, expecting an error.
\end{itemize}

\subsection*{b. \texttt{DoctorsAPITestCase}}
This test case validates doctor-related endpoints.

\textbf{Setup:}
\begin{itemize}
    \item A \texttt{Doctor} object is created with attributes like \texttt{specialization}, \texttt{availability}, and \texttt{about}.
\end{itemize}

\textbf{Tests:}
\begin{itemize}
    \item \texttt{test\_get\_doctor\_by\_id}: Fetches a doctor by ID and verifies their details.
    \item \texttt{test\_get\_doctors\_by\_specialization}: Retrieves doctors by specialization and ensures the correct subset is returned.
    \item \texttt{test\_get\_all\_doctors}: Fetches all doctors and validates the total count.
    \item \texttt{test\_post\_doctor}: Creates a new doctor with a POST request.
    \item \texttt{test\_put\_doctor}: Updates an existing doctor's details using a PUT request.
    \item \texttt{test\_patch\_doctor}: Partially updates a doctor's availability with a PATCH request.
    \item \texttt{test\_delete\_doctor}: Deletes a doctor using a DELETE request.
    \item \texttt{test\_get\_doctors\_with\_no\_free\_specialists}: Tests when no free specialists are available for a given specialization.
\end{itemize}

\subsection*{c. \texttt{NewUsersAPITestCase}}
This test case validates general user-related operations.

\textbf{Setup:}
\begin{itemize}
    \item Two \texttt{NewUser} objects are created for testing.
\end{itemize}

\textbf{Tests:}
\begin{itemize}
    \item \texttt{test\_get\_user\_by\_id}: Retrieves a user by ID.
    \item \texttt{test\_get\_nonexistent\_user\_by\_id}: Tests querying a non-existent user.
    \item \texttt{test\_get\_all\_users}: Fetches all users and validates the total count.
    \item \texttt{test\_post\_new\_user}: Tests the creation of a new user.
    \item \texttt{test\_put\_update\_user}: Updates user details with a PUT request.
    \item \texttt{test\_patch\_update\_user}: Partially updates a user's email using PATCH.
    \item \texttt{test\_delete\_user}: Deletes a user.
    \item \texttt{test\_delete\_nonexistent\_user}: Tests deleting a non-existent user.
\end{itemize}

\subsection*{d. \texttt{IntermediateAPITestCase}}
This test case manages intermediate users (e.g., admins, coordinators).

\textbf{Setup:}
\begin{itemize}
    \item A sample intermediate user is created.
    \item \texttt{valid\_data} and \texttt{invalid\_data} dictionaries are prepared.
\end{itemize}

\textbf{Tests:}
\begin{itemize}
    \item \texttt{test\_get\_all\_intermediates}: Retrieves all intermediate users.
    \item \texttt{test\_get\_single\_intermediate\_valid\_id}: Fetches an intermediate user by a valid ID.
    \item \texttt{test\_get\_single\_intermediate\_invalid\_id}: Tests querying a non-existent intermediate user.
    \item \texttt{test\_post\_valid\_intermediate}: Creates a new intermediate user.
    \item \texttt{test\_post\_invalid\_intermediate}: Ensures validation errors for invalid data.
    \item \texttt{test\_patch\_intermediate}: Partially updates an intermediate user's \texttt{about} field.
    \item \texttt{test\_delete\_intermediate}: Deletes an intermediate user.
    \item \texttt{test\_delete\_invalid\_intermediate}: Tests deleting a non-existent intermediate user.
\end{itemize}

\subsection*{e. \texttt{AppointmentAPITestCase}}
This test case focuses on appointment management endpoints.

\textbf{Setup:}
\begin{itemize}
    \item Two \texttt{Appointment} objects are created.
\end{itemize}

\textbf{Tests:}
\begin{itemize}
    \item \texttt{test\_get\_all\_appointments}: Fetches all appointments.
    \item \texttt{test\_get\_filtered\_appointments}: Filters appointments by \texttt{meeting\_Type}.
    \item \texttt{test\_create\_valid\_appointment}: Creates a new appointment with valid data.
    \item \texttt{test\_update\_appointment}: Updates an appointment's \texttt{disease} field.
    \item \texttt{test\_delete\_appointment}: Deletes an appointment.
\end{itemize}

\section*{3. Key Features Tested}
\begin{itemize}
    \item CRUD operations.
    \item Validation for invalid data or non-existent IDs.
    \item Filtering results by attributes.
    \item Secure password hashing.
\end{itemize}

\section*{4. Improvements Suggested}
\begin{itemize}
    \item Uncomment and refine commented tests.
    \item Add assertions for response status codes.
    \item Enhance validation tests for a broader range of inputs.
    \item Include tests for user permissions and authentication.
\end{itemize}

\section*{5. Conclusion}
The test suite is robust, leveraging DRF's \texttt{APITestCase} for comprehensive API endpoint coverage. Each test ensures correctness, reliability, and resilience of the backend.

\begin{figure}[h]
    \centering
    \includegraphics[width=\textwidth]{SE Testing ss.jpg}
    \caption{. =$>$ SUCCESS, F =$>$ FAILURE, E =$>$ ERROR}
    \label{fig:example_image}
\end{figure}

\end{document}